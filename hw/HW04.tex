\documentclass[12pt]{article}
\usepackage[left=2cm,right=2cm,top=2cm,bottom=2cm,bindingoffset=0cm]{geometry}
\usepackage[utf8x]{inputenc}
\usepackage[english,russian]{babel}
\usepackage{cmap}
\usepackage{amssymb}
\usepackage{amsmath}
\usepackage{url}
\usepackage{pifont}
\usepackage{tikz}
\usepackage{verbatim}

\usetikzlibrary{shapes,arrows}
\usetikzlibrary{positioning,automata}
\tikzset{every state/.style={minimum size=0.2cm},
initial text={}
}


\newenvironment{myauto}[1][3]
{
  \begin{center}
    \begin{tikzpicture}[> = stealth,node distance=#1cm, on grid, very thick]
}
{
    \end{tikzpicture}
  \end{center}
}


\begin{document}
\begin{center} {\LARGE Формальные языки} \end{center}

\begin{center} \Large домашнее задание до 23:59 16.03 \end{center}
\bigskip

\begin{enumerate}
  \item Доказать или опровергнуть свойство регулярных выражений:
  \[
    \forall p, q \text{ --- регулярные выражения}: (p \mid q)^* = p^*(qp^*)^*
  \]
  \item Доказать или опровергнуть свойство регулярных выражений:
  \[
    \forall p, q \text{ --- регулярные выражения}: (p q)^* p = p (q p)^*
  \]
  Решение: Представленный ниже автомат подходит для обоих выражений.
  \begin{myauto}
    \node[state,initial]   (q_0) {$q_0$};
     \node[state, accepting] (q_1) [right=of q_0] {$q_1$};

    \path[->] (q_0) edge [bend left]             node [above] {$p$}    (q_1)
    		  (q_1) edge [bend left]             node [below] {$q$} (q_0)

              
    ;
  \end{myauto}
  \item Доказать или опровергнуть свойство регулярных выражений:
  \[
    \forall p, q \text{ --- регулярные выражения}: (p q)^* = p^* q^*
  \]
  Решение:
  Строка $pqpq$ будет распознона первым выражением, но не вторым 
  \item Для регулярного выражения:
   \[ (a \mid b)^+ (aa \mid bb \mid abab \mid baba)^* (a \mid b)^+\]
  Построить эквивалентные:
  \begin{enumerate}
    \item Недетерминированный конечный автомат
    \item Недетерминированный конечный автомат без $\varepsilon$-переходов
    \item Минимальный полный детерминированный конечный автомат
  \end{enumerate}

\end{enumerate}

\end{document}
